\documentclass[a4paper]{article}
\usepackage[left=3.5cm, right=2.5cm, top=2.5cm, bottom=2.5cm]{geometry}
\usepackage[MeX]{polski}
\usepackage[utf8]{inputenc}
\usepackage{graphicx}
\usepackage{enumerate}
\usepackage{amsmath} %pakiet matematyczny
\usepackage{amssymb} %pakiet dodatkowych symboli
\usepackage[ampersand]{easylist}

\title{Dokument LaTeX programy uzytkowe}
\author{Marcel Paluch}
\date{15 Października 2022}
\begin{document}
\maketitle
\newpage
\section{sekcja numer jeden, tutaj pisze jakies bzdury zeby bylo widac, ze zrobilem sekcje}
\subsection{a tutaj bardzo prosze jest podsekcja powyzszej sekcji}
\subsection{subsekcja 2}
\subsection{subsekcja 3}
\subsubsection{a tu nawet jest sub sub sekcja}
\newpage
\paragraph{paragraf numer jeden ciekawe co w nim bedzie}
\subparagraph{podparagraf paragrafu numer jeden }
\subparagraph{a tu paragraf zagniezdzony w glownym paragrafie \newline }
\appendix{ciekawe co sie zadzieje jak dodamy tu appendix}
\newpage
\begin{enumerate}[A)]
\item punkt pierwszy
\item punkt drugi
	\begin{enumerate}[a.]
	\item ciekawe czy zadziala
	\item kolejny zagniezdzony item
	\item i jeszcze kolejny
	\end{enumerate}
\item a tu wyliczane itemy juz poza zagniezdzeniem
\end{enumerate}
\newpage
\begin{center}
{\huge \textbf{Racuchy na jogurcie}} \newline
\end{center}
\textbf{Racuchy na jogurcie} z dużą ilością jabłek, to pyszny pomysł nie tylko na śniadanie, ale i na kolację. Przepis nie zawiera drożdży, a do tego aby je zrobić nie potrzebujesz nawet miksera. \newline
\ListProperties (Hide=100, Progressive=3ex, Style*=--)
\begin{easylist}
& jabłka można zastąpić innymi owocami
& tylko kilka popularnych składników
& bez drożdży i bez miksera \newline
\end{easylist}

\noindent\textbf{Czas przygotowania}: 35 minut \newline
\textbf{Liczba porcji}: 16 dużych racuchów \newline \\
\textbf{Kaloryczność kcal}: 180 w 1 racuchu \newline
\textbf{Dieta}: wegetariańska \newline \\
\textbf{Składniki}: 
\begin{itemize}
\item 2 szklanki mąki pszennej uniwersalnej - 320 g
\item 400 g jogurtu naturalnego - duży kubełek
\item 3 średnie lub większe jajka - około 170 g po rozbiciu
\item 4 łyżki cukru - około 60 g
\item 4 łyżki oleju do smażenia lub roztopionego masła
\item po jednej łyżeczce sody i proszku do pieczenia
\item 2 średniej wielkości jabłka - około 400 g
\item olej lub masło klarowane do smażenia placków
\end{itemize}
\large{Racuszki na jogurcie} \newline
Szklanka ma u mnie pojemność 250 ml. \newline
Jajka oraz jogurt muszą być wcześniej wyjęte z lodówki, by osiągnęły temperaturę pomieszczenia. W tym przepisie najważniejsze jest to, by składniki nie były zimne. \newline \\
Kalorie policzone zostały na podstawie użytych przeze mnie składników. Jest to więc orientacyjna liczba kalorii, ponieważ Twoje składniki mogą mieć inną liczbę kalorii niż te, których użyłam ja. Z podanej ilości składników wyszło mi 16 grubych racuszków o średnicy około 10-12 cm. \newline \\
Jabłka ważone były przed obraniem. Zawsze podaję wagę użytych przeze mnie warzyw lub owoców. Nie trzeba jednak stosować się do wytycznych co do grama. Wagi podawane są po to, by łatwiej Wam było zorientować się, jaka mniej więcej ilość potrzebna jest do zrobienia danego dania. 
\end{document}
